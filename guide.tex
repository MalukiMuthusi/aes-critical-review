%% read and take notes about AES (Advance Encryption Standard)

% What kind of article is it
% What is the main area under discussion?
% What are the main findings?
% What are the stated limitations?
% Where does the author's data and evidence come from
% What are the main issues raised by the author
% What questions are raised?
% How well are these questions addressed
% What are the major points/interpretations made by the author in terms of the issues raised?
% Is the text balanced? Is it fair / biased?
% Does the author contradict herself?
% How does all this relate to other literature on this topic?
% How does all this relate to your own experience, ideas and views?
% What else has this author written? Do these build / complement this text?
% Has anyone else reviewed this article? What did they say? Do I agree with them?

%% Organising your writing

% You first need to summarise the text that you have read. One reason to summarise the text is that the reader may not have read the text.
% Why is this topic important?
% Where can this text be located? For example, does it address policy studies?
% What other prominent authors also write about this?

%% Evaluation

% Evaluation is the most important part in a critical review.
% Use the literature to support your views. You may also use your knowledge of conducting research, and your own experience. Evaluation can be explicit or implicit.


% Explicit evaluation
% Explicit evaluation involves stating directly (explicitly) how you intend to evaluate the text.

% e.g. "I will review this article by focusing on the following questions. First, I will examine the extent to which the authors contribute to current thought on Second Language Acquisition (SLA) pedagogy. After that, I will analyse whether the authors' propositions are feasible within overseas SLA classrooms."

% Implicit evaluation
% Implicit evaluation is less direct. The following section on Linguistic Features of Writing a Critical Review contains language that evaluates the text.

%% Linguistic features of a critical review
% The following examples come from published critical reviews. Some of them have been adapted for student use.

% Summary Language

% Evaluation Language

% Conclusion Language

% https://www.ucl.ac.uk/ioe-writing-centre/critical-reading-and-writing/critical-review